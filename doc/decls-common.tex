%% Generic environments and commands that we use in the CS:APP book

%%%%%%%%%% 
% Commands
%%%%%%%%%% 

% Reference to versions of book
\newcommand{\csapp}{CS:APP}
\newcommand{\csappii}{CS:APP2e}

% Reference to something in book from a web aside
\newcommand{\bookref}[1]{CS:APP3e-\ref{#1}}
% Reference to a named web aside
\newcommand{\wasideref}[1]{\textsc{#1}}

% For drawing different kinds of circles in the course table in the preface
% fcirc: filled circle, % ecirc: empty circle
\newcommand{\fcirc}{\begin{picture}(1,1)\put(1,1){\circle*{5}}\end{picture}}
\newcommand{\ecirc}{\begin{picture}(1,1)\put(1,1){\circle{5}}\put(1,1){\circle*{2.4}}\end{picture}}

% For putting labels on pictures
\newcommand{\clabel}[1]{\makebox(0,0){#1}}
\newcommand{\rlabel}[1]{\makebox(0,0)[r]{#1}}
\newcommand{\llabel}[1]{\makebox(0,0)[l]{#1}}
\newcommand{\blabel}[1]{\makebox(0,0)[b]{#1}}
\newcommand{\lblabel}[1]{\makebox(0,0)[lb]{#1}}
\newcommand{\tlabel}[1]{\makebox(0,0)[t]{#1}}

%% Comment command
\newcommand{\comment}[1]{}

%% Two part captions.  First part is title.  Second is explanation.
\newcommand{\mycaption}[2]{\caption[#1]{\small \textsf{\textbf{#1} #2}}}

%% For showing categories of homework problems
\definecolor{catcolor}{rgb}{0,0,1}
\newcommand{\catmark}{{\color{catcolor}\blacklozenge}}
\newcommand{\catone}{$\catmark$}
\newcommand{\cattwo}{$\catmark\,\catmark$}
\newcommand{\catthree}{$\catmark\,\catmark\,\catmark$}
\newcommand{\catfour}{$\catmark\,\catmark\,\catmark\,\catmark$}
\newcommand{\showcat}[1]{\ifthenelse{#1=1}{\catone}{%
 \ifthenelse{#1=2}{\cattwo}{%
   \ifthenelse{#1=3}{\catthree}{\catfour}}}}


% Linker ref->def notation
\newcommand{\refdef}[2]{$\texttt{REF(#1)} \rightarrow \texttt{DEF(#2)}$}

% Linker ref->def question
\newcommand{\refdefq}[1]{$\texttt{REF(#1)} \rightarrow \texttt{DEF(\_\_\_\_\_\_\_\_\_\_.\_\_\_\_)}$}

%%%%%%%%%%%%%%%%%%%%%%%%%%%%%%%%%%%%%%%%%%%%%%%%%%%%%%%%%%%% 
% Environments and macros for different elements of the book
%%%%%%%%%%%%%%%%%%%%%%%%%%%%%%%%%%%%%%%%%%%%%%%%%%%%%%%%%%%%% 

%% This TeX counter is used to generate references to 
%% different code line numbers
\newcounter{codelineno}

%% This defines the color we are using for all the code line numbers
\definecolor{linenumcolor}{rgb}{0,0,1} 

%% Various Unix command-line prompts
\newcommand{\unixprompt}{linux>}
\newcommand{\linuxprompt}{linux>}
\newcommand{\solarisprompt}{solaris>}

%% Definition of RIO, SIO  package names
\newcommand{\rio}{\textsc{Rio}}
\newcommand{\sio}{\textsc{Sio}}
\newcommand{\Tiny}{\textsc{Tiny}}

%%%
%%% Environments
%%%

%% pcode- Generating pseudo-code
\newenvironment{pcode}{\mbox{}\newline\vspace{.5ex}\newline\begin{minipage}{6in}\begin{tt}\begin{tabbing}MMM\=MMM\=MMM\=MMM\=MMM\=MMM\=MMM\=MMM\=MMM\=\+\kill}{\end{tabbing}\end{tt}\end{minipage}\newline\vspace{.5ex}\newline}

%% Showing syntactic elements in pseudo-code
\newcommand{\syntax}[1]{{\rm\it #1}}

%% aside - for displaying sidebars 
\newenvironment{aside}[1]%
{\begin{quote}\footnotesize{\textbf{Aside: #1}}\\}
{\textbf{End Aside.}\end{quote}}

%% waside - for highlighting content available on the WEE
%% Arg1: Code.  Arg2: Title
\newenvironment{waside}[2]%
{\begin{quote}\footnotesize{\textbf{Web Aside \textsc{#1}: #2}}\\}
{\textbf{End Aside.}\end{quote}}

%% ntc - New to C element
\newenvironment{ntc}[1]%
{\begin{quote}\footnotesize{\textbf{New to C?: #1}}\\}
{\textbf{End.}\end{quote}}

%% principle - to highlight key concept
\newenvironment{principle}[1]%
{\begin{quote}\textbf{Principle}: {\em #1} \hrulefill \mbox{} \vskip 1pt}%
{\vskip 1pt \hrulefill \mbox{} \end{quote}}

%% derivation - to present derivation
\newenvironment{derivation}[1]%
{\begin{quote}\textbf{Derivation}: {\em #1} \hrulefill \mbox{} \vskip 1pt}%
{\vskip 1pt \hrulefill \mbox{} \end{quote}}

%% ccode- for displaying formatted C code (c2tex) 
\newenvironment{ccode}%
{\small}%
{}

%% scode - for displaying formatted ASM code (s2tex and d2tex)
\newenvironment{scode}%
{\small}
{}

%% codefrag - for displaying unformatted code fragments
\newenvironment{codefrag}%
{\small\begin{alltt}}%
{\end{alltt}%
}

%% tty - for displaying TTY input and output
\newenvironment{tty}%
{\small\begin{alltt}}%
{\end{alltt}}


%% Environment for descriptions that provide aligned text
\newenvironment{mydesc}[1]{
\setbox1=\hbox{#1}
\begin{list}{}{
\setlength{\labelwidth}{\wd1}
\setlength{\leftmargin}{\wd1}
  \addtolength{\leftmargin}{1em}
  \addtolength{\leftmargin}{\labelsep}
\setlength{\rightmargin}{1em}}}{\end{list}}

\newcommand{\litem}[1]{\item[#1\hfill]}
\newcommand{\ritem}[1]{\item[#1]}

%% For marking different verions of problems

\newcommand{\sameprob}[1]{\textbf{Same as Version A Problem \ref{#1}}\\ \mbox{}}

%%%%%%%%%%%%%%%%%%%%%%%%%%%%%%%%%%%%%%%%%%%%%%%%%%%%%%%%%%%%%%%%%%%%%%%%%%%%
%% Indexing 
%%%%%%%%%%%%%%%%%%%%%%%%%%%%%%%%%%%%%%%%%%%%%%%%%%%%%%%%%%%%%%%%%%%%%%%%%%%%
%% Distinguish multiple types of index entries
%%   - IA32 instructions (iindex, diindex)
%%   - x86-64 instructions (xindex, dxindex)
%%   - Y86  instructions (yindex, dyindex)
%%   - HCL  things (hindex, dhindex)
%%   - Program names (pindex, dpindex)
%%   - C things (cindex, dcindex)
%%   - Stdlib functions (sindex, dsindex)
%%   - Unix functions (uindex, duindex)
%%   - CSAPP functions (csindex, dcsindex)
%%   - CSAPP programs (pcsindex, dpcsindex)
%%   - Library files, such as <limits.h> (lindex, dlindex) 
%%   - Everything else. (index, dindex)
%% In every case, the `d' version is used for the defining index entry

%% Generic instructions.
%% Takes two arguments:
%% The name of the instruction
%% An English rendition of the mnemonic
\newcommand{\gindex}[2]{\index{#1@\textsc{#1} [Instruction class] #2}}
\newcommand{\dgindex}[2]{\index{#1@\textsc{#1} [Instruction class] #2|emph}}


%% IA32 instructions.
%% Takes two arguments:
%% The name of the instruction
%% An English rendition of the mnemonic
\newcommand{\iindex}[2]{\index{#1@\texttt{#1} [IA32] #2}}
\newcommand{\diindex}[2]{\index{#1@\texttt{#1} [IA32] #2|emph}}

%% x86-64 instructions.
%% Takes two arguments:
%% The name of the instruction
%% An English rendition of the mnemonic
\newcommand{\xindex}[2]{\index{#1@\texttt{#1} [x86-64] #2}}
\newcommand{\dxindex}[2]{\index{#1@\texttt{#1} [x86-64] #2|emph}}

%% y86-64 instructions.
%% Takes two arguments:
%% The name of the instruction
%% An English rendition of the mnemonic
\newcommand{\yindex}[2]{\index{#1@\texttt{#1} [Y86-64] #2}}
\newcommand{\dyindex}[2]{\index{#1@\texttt{#1} [Y86-64] #2|emph}}

%% HCL things
%% Takes two arguments:
%% The name of the instruction
%% An English rendition of the mnemonic
\newcommand{\hindex}[2]{\index{#1@\texttt{#1} [HCL] #2}}
\newcommand{\dhindex}[2]{\index{#1@\texttt{#1} [HCL] #2|emph}}

%% Program names (e.g., gcc, gdb.  Write in lower case)
%% Takes two arguments:
%% The name of the program
%% An English description
\newcommand{\pindex}[2]{\index{#1@{\sc #1} #2}}
\newcommand{\dpindex}[2]{\index{#1@{\sc #1} #2|emph}}

%% C things (operators, statements, etc)
%% Takes two arguments:
%% C name
%% An English description
\newcommand{\cindex}[2]{\index{#1@\texttt{#1} [C] #2}}
\newcommand{\dcindex}[2]{\index{#1@\texttt{#1} [C] #2|emph}}

%% Stdlib functions
%% Takes two arguments:
%% function name
%% An English description
\newcommand{\sindex}[2]{\index{#1@\texttt{#1} [C Stdlib] #2}}
\newcommand{\dsindex}[2]{\index{#1@\texttt{#1} [C Stdlib] #2|emph}}

%% Unix functions
%% Takes two arguments:
%% function name
%% An English description
\newcommand{\uindex}[2]{\index{#1@\texttt{#1} [Unix] #2}}
\newcommand{\duindex}[2]{\index{#1@\texttt{#1} [Unix] #2|emph}}

%% CSAPP functions
%% Takes two arguments:
%% function name
%% An English description
\newcommand{\csindex}[2]{\index{#1@\texttt{#1} [CS:APP] #2}}
\newcommand{\dcsindex}[2]{\index{#1@\texttt{#1} [CS:APP] #2|emph}}

%% CSAPP programs
%% Takes two arguments:
%% function name
%% An English description
\newcommand{\pcsindex}[2]{\index{#1@{\sc #1} [CS:APP] #2}}
\newcommand{\dpcsindex}[2]{\index{#1@{\sc #1} [CS:APP] #2|emph}}

%% Library file names (e.g., <limits.h>)
%% Takes two arguments:
%% The name of the file (without brackets)
%% An English description
\newcommand{\lindex}[2]{\index{#1@\texttt{<#1>} #2}}
\newcommand{\dlindex}[2]{\index{#1@\texttt{<#1>} #2|emph}}

%% Normal index entry.
%% Takes one argument
%% \index already defined
\newcommand{\dindex}[1]{\index{#1|emph}}



